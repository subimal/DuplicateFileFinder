\chapter{The Classes and Functions}

\section{The classes}

\subsection{FileObject}
This class defines an object with the following attributes:
\paragraph{\bf name} It contains the file name of a {\tt FileObject}
\paragraph{\bf md5hex} It contains the MD5 hex digest of the file. Initial value is {\tt None}. Its value is set through the {\tt hash()} attribute by invoking the function {\tt GetMD5Sum}.
\paragraph{\bf hash()} This function will generate the MD5 checksum hexdigest (if the {\bf md5hex} is {\tt None}) by invoking the {\tt GetMD5Sum} function. \marginpar{\tiny What if the file was modified after the MD5 sum was generated? Compare the time when md5hex was assigned and the time of last access of the file?}

\section{The Functions}

\subsection{GetMD5Sum}
\subsubsection{Arguments}
\paragraph{\tt filename} A {\it string} containing file's name including the full path.
\paragraph{\tt chunksize} (Optional). The buffer size for reading the file in parts. Default value is 25600 bytes.

\subsubsection{Return value}
The MD5 hexdigest for the file.

\subsection{FileList}

\section 
